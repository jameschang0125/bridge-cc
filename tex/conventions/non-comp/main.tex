\section{non-contested conventions}

\subsection{Gazzilli accepted}

After Gazilli established GF sequence, we have several things to achieve:
consider the simpler case of \texttt{1S - 1N} and \texttt{1H - 1S}. 
Since both are capped, we want to:

\begin{itemize}
    \setlength\itemsep{0pt}
    \item investigate major fit to choose the correct game
    \item if someone have shape, there may be slam (or 5m)
    \item if both don't have shape, but both max, there may be slam
\end{itemize}

To address these and still maintaining as natural as possible, I suggest:
\begin{itemize}
    \setlength\itemsep{0pt}
    \item if opener have shape (5-5 or 4oM for 1M opening), just bid it
    \item otherwise, opener have a cheap catchall bid indicating "I don't have shape" (as you can see later, we put it in the cheapest bid and exchange the original natural meaning with 2N)
    \item after opener's catchall, resp's bid under 2N all indicates "I don't have shape" with natrual meaning.
    \item otherwise, resp will bid his shape
    \item after both catchalls, if opener is maximum, he can bid a suit. Otherwise, he can simply CoG by 3N or rebidding his suit
\end{itemize}

This is the "double waiting" design I propose:


\begin{bidsemi}
\bid{1H - 1S; 2C - 2D}
\bid{2H}[weak variant: 6+H]
\bid{2S}[waiting]
    \subbid{2N}[waiting]
        \subsubbid{3m}[max, 4+m]
        \subsubbid{3H}[6+H, not promising extra]
    \subbid{3m}[max, 5+m; or 6+m]
\bid{2N}[4+S]
\bid{3m}[5+m]
\bid{3H}[solid 6+H]
\end{bidsemi}
\begin{bidsemi}
\bid{1H - 1N; 2C - 2D}
\bid{2H}[weak variant: 6+H]
\bid{2S}[waiting]
    \subbid{2N}[waiting]
        \subsubbid{3m}[max, 4+m]
        \subsubbid{3H}[6+H, not promising extra]
        \subsubbid{3S}[2S, not promising extra]
    \subbid{3m}[11+, 4+m; or 5+m (note: because now resp is uncap but unlikely to have 5m)]
\bid{2N}[3+S]
\bid{3m}[5+m]
\bid{3H}[solid 6+H]
\end{bidsemi}
\begin{bidsemi}
\bid{1S - 1N; 2C - 2D}
\bid{2H}[waiting]
    \subbid{2S/2N}[waiting with 2S/1-S]
        \subsubbid{3m}[max, 4+m]
        \subsubbid{3H}[3H, not promising extra]
        \subsubbid{3S}[6+S, not promising extra]
\subbid{3m}[max, 5+m; or 6+m]
\bid{2S}[weak variant: 6+S]
\bid{2N}[4+H]
\bid{3X}[5+X]
\bid{3S}[solid 6+S]
\end{bidsemi}

Given these principles, it is easy to extend to minor Gazilli:

\begin{bidsemi}
\bid{1D - 1M; 1N - 2C}
\bid{2D}[weak variant: min unbal]
\bid{2oM}[waiting]
    \subbid{2M/2N}[nat waiting]
        \subsubbid{3m}[6+m, not promising extra]
        \subsubbid{3M}[3M, not promising extra]
        \subsubbid{new suit}[nat extra]
    \subbid{3m}[nat extra]
    \subbid{3M}[6+M, not promising extra]
\bid{2N}[indicates oM, usually 6D-4oM]
\bid{3m}[6+m, extra]
\bid{new suit}[nat, shapely]
\end{bidsemi}
\begin{bidsemi}
\bid{1C - 1DH; 1S - 1N}
\bid{2C}[weak variant: min unbal]
\bid{2D}[waiting]
    \subbid{2M/2N}[nat waiting]
        \subsubbid{3m}[6+m, not promising extra]
        \subsubbid{3M}[3M, not promising extra]
        \subsubbid{new suit}[nat extra]
    \subbid{3m}[nat extra]
    \subbid{3M}[6+M, not promising extra]
\bid{2N}[indicates D, usually 6m-4oM and maybe extra (D is not as important as M)]
\bid{3m}[6+m, extra]
\bid{new suit}[nat, shapely]
\end{bidsemi}

\subsection{modified 2-way}

Two way is ON only after \texttt{1C - (P/X) - 1DH; 1N} or \texttt{1C - (1D) - X; 1N}.

\bin{\currfiledir/seq/2way}

For \texttt{1C - 1S; 1N}: default off, but we have \texttt{2C -> 2D} and \texttt{2D} inv.

\subsection{PLOB (4SF1)}



