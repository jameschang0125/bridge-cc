\section{Summary}

\subsection{general style}

\bin{\currfiledir/seq/style}

\subsection{opening summary}

\bin{\currfiledir/seq/sum}



