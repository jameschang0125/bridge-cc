\section{vs artificial bids}

Double is natrual (or lead-directing) unless otherwise specified.

\subsection{vs transfer}

A \texttt{transfer opening (preempt)} is defined as:

\begin{itemize}
    \setlength{\itemsep}{0pt}
    \item The bid promises one \textbf{specific} higher-ranking suit
    \item It also counts even if have a strong variant that is not the one promised. ex: \texttt{2D = weak H or strong S+C} is a transfer opening; but \texttt{2D = weak H or weak S} is not.
\end{itemize}

Against transfer openings, since we are allowed to double almost freely, and we have four ways of showing strengths: \texttt{X then P/X}, \texttt{P then X}, and \texttt{direct cuebid}.
However, there are some questions: first, how do we want to seperate our hands (or even the more primitive question: what hands do we want to "make a call") into these groups?
Second, what is a reasonable assignment? \\
To answer these questions, let's see how these calls are different:
\begin{itemize}
    \setlength{\itemsep}{0pt}
    \item X then P/X: resp are allowed to make a move after the first X, so these two hands should share a trait indicating "resp can move"
    \item P then X: resp usually won't move before the second X unless he have t/o strength
    \item direct cuebid: resp is forced to bid
\end{itemize}

My suggestions are:

{\small\ttfamily
\begin{NiceTabular}{ccccc}[hvlines]
       & strength & takeout & allow penalty & penalty \\
X -> P & weak*    & yes     &               &         \\
X -> X & normal   & yes     & yes           &         \\
P -> X & penalty  &         &               & yes     \\
cue    & normal   & yes     &               &         \\
\end{NiceTabular}}

{\small\ttfamily*originally couldn't takeout, or t/o minimum. So roughly 9-12/10-13/12-15 at 1/2/3-level} \\

The bidding usually goes like \texttt{(3C*) - X - (3D)} to the responder.
He simply replies assuming the overcaller is a weaker takeout hand.
If he passes and sees the overcaller doubles again, he would assume the overcaller is a normal takeout hand but with 2-3(4) cards in opponent's suit
(with 0-1 cards and the same strength, he will cuebid instead), and is allowed to bid normally or pass to convert to penalty. \\
What do we gain? First, we allow some weaker takeout hands to bid, especially when against a high-level preempt.
Second, for normal takeout hands, instead of mixing into \texttt{(3D) - X}, we now allows a real penalty hand to penalize, and a balanced takeout hand to cooperatively penalize.

\subsubsection{delayed bid}

Let's just assume pass then bid is weaker for now.

\subsection{vs artificial raise}

\begin{itemize}
    \setlength{\itemsep}{0pt}
    \item vs Drury: t/o
    \item vs Bergen: t/o
    \item vs splinter: suggest sacrifice
\end{itemize}


