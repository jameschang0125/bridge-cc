\usepackage{amsmath}
\usepackage{amsthm}
\usepackage{amssymb}
\usepackage{amsfonts}
\usepackage{mathrsfs}
\usepackage{mathtools}

% package for symbols
\usepackage{stmaryrd}
\usepackage{halloweenmath}
\usepackage{wasysym}
\usepackage{ccicons}
\usepackage{fge}

\usepackage{hyperref}
\usepackage{cleveref}
\hypersetup{
  colorlinks,
  citecolor=violet,
  linkcolor=blue,
  urlcolor=blue
}

\usepackage[most]{tcolorbox}
\newtcbtheorem{Theorem}{Theorem}{
  enhanced,
  sharp corners,
  attach boxed title to top left={
    yshifttext=-1mm
  },
  colback=white,
  colframe=blue!75!black,
  fonttitle=\bfseries,
  boxed title style={
    sharp corners,
    size=small,
    colback=blue!75!black,
    colframe=blue!75!black,
  } 
}{thm}
\newtcbtheorem{Problem}{Problem}{
  enhanced,
  sharp corners,
  attach boxed title to top left={
    yshifttext=-1mm
  },
  colback=white,
  colframe=blue!25,
  fonttitle=\bfseries,
  coltitle=black,
  boxed title style={
    sharp corners,
    size=small,
    colback=blue!25,
    colframe=blue!25,
  } 
}{prob}

%=======cool theorem box tutorial=======%
\begin{comment}
\begin{Theorem}{}{fermat}
  No three positive integers \(a\), \(b\) and \(c\) satisfy the equation \(a^{n}
  + b^{n} = c^{n}\) for any integer greater than or equal to two.
\end{Theorem}
\end{comment}

\begin{comment}
\begin{Problem}{A slightly difficult problem*}{}~
\begin{center}
\includegraphics[height=40mm]{lieb.png}
%\caption{this is caption}
\label{lieb}
\end{center}
Try to perform the measures above.
\end{Problem}
\end{comment}













% defines
\DeclareMathOperator*{\argmax}{arg\!\max}
\DeclareMathOperator*{\argmin}{arg\!\min}
\newcommand{\C}{\mathbb{C}}
\newcommand{\R}{\mathbb{R}}
\newcommand{\Q}{\mathbb{Q}}
\newcommand{\Z}{\mathbb{Z}}
\newcommand{\N}{\mathbb{N}}
\newcommand{\w}{\mathbf{w}}
\newcommand{\x}{\mathbf{x}}
\newcommand{\y}{\mathbf{y}}
\renewcommand{\v}{\mathbf{v}}
\newcommand{\E}{\mathbb{E}}
\newcommand{\norm}[1]{{\left\lVert#1\right\rVert}}
\newcommand{\floor}[1]{{\left\lfloor#1\right\rfloor}}
\newcommand{\ceil}[1]{{\left\lceil#1\right\rceil}}
\newcommand{\abs}[1]{{\left|#1\right|}}
\theoremstyle{definition}
\newtheorem{answer}{}
\newtheorem{problem}{Problem}
\newtheorem{theorem}{Theorem}[section]
\newtheorem{definition}[theorem]{Definition}
\newtheorem{property}[theorem]{Property}
\newtheorem{corollary}{Corollary}[theorem]
\newtheorem{lemma}{Lemma}



