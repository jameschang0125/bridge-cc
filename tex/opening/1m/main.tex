\section{1m}

% \bin{\currfiledir/seq/1m}
Again, thanks to the space preserved by dutch doubleton and transfer responses, one can see that \texttt{1C - 1DH - 1S} and \texttt{1D - 1HS - 1N} (since no min bal) is undefined.
Therefore we are allowed to use them to show strong (16+) hands, with many higher bidding spaces left.
For example, after \texttt{1D - 1S}, since \texttt{1N} shows all strong (16+) hands, \texttt{2H, 3C, 3H} are undefined.
Unlike major suit openings where the opener can easily have shapely hands, minors don't (and among those does, lot of them are single-suited).
Therefore, we decide to prioritize major fits -- using jump rebids as mini-splinter.
For normal reverses, since we have already dealt with mini-splinters, we simply leave it natrual "with-Gazilli-style": showing a shapely (6-4) hand with 14-16 concentrated HCP.


Similar to major suit Gazzilli, we are allowed to include a weak variant by rebidding \texttt{2X} after Gazzilli accepted (i.e. responder rebids \texttt{+1}).
In natrual systems, however, we usually rebid \texttt{2m} with any weak unbal hands. Therefore, we are left two seemingly unnessesary options: direct rebid of \texttt{2m} and Gazzilli then \texttt{2m}.
There may be several options for this: for example, identifying a 3-card fit (compare to \texttt{1m - 1M - 2M} may be 3-card or \texttt{1C - 1DH - 1HS} is (2)3-card); or perhaps game try (targeting \texttt{3N}) with a good (AJ9 or KQ) 6+ suit.
In our system, we choose to do BOTH (obviously, slightly weaker then choosing one) by agreeing:
\begin{itemize}
  \setlength\itemsep{0pt}
  \item direct rebid \texttt{2m}: 14-15, good 6+ suit
  \item jump rebid \texttt{3m}: 16-17, good 6+ suit
  \item min unbal uses the Gazzilli. If accepted, rebid \texttt{2M \& 2m} to show min unbal with or without 3-card fit
\end{itemize}
Note that there are a few side effects: first, this also tighten the range of the original \texttt{3m} rebid (from 15-17 to 16-17);
second, rebidding \texttt{3m} becomes game-forcing (18+);
last, strong hands cannot show 3-card fit using \texttt{2M} (note: only after Gazzilli accepted), but we don't think it's a big deal since there are plenty of spaces left, including seemingly undefined \texttt{2N}.

\subsection{resp}

The most noticable differences are transfer responses and jumps.
\textcolor{lightgrey}{(IMHO, weak jumps and splinters to \texttt{1m} are rarely efficient)}
Another change (recommended by Jonky) is the "reverse Flannery" which shows 3-7 HCP and 54xx+ (usually 5-7 but can be weaker due to length or Vul), therefore \texttt{1S} response followed by \texttt{2H} shows 8+.
This synergized quite well with minor-suit Gazzilli because we are allowed to show a constructed (8-10) 54xx+ with \texttt{1C - 1H; 1S - 2H \& 1D - 1S; 1N - 2H}.

\bin{\currfiledir/seq/resp}

\subsection{rebid}

We have described most rebids previously. For subsequent auctions, we simply use natrual (jump = inv, 4SF, new-suit F) except PLOB and modified 2-way.


\begin{bidsemi}
    \bid{1m - 1M(-1)}
    \bid{2N}[17-18 bal]
    \bid{2m}[(13)14-15, good (two of AKQ) 6+m]
        \subbid{rebid}[nat F1]
        \subbid{raise \& new suit}[nat GF]
    \bid{3m}[good (two of AKQ) 6+m (could be weaker with longer m)]
    \bid{3M}[(16)17-18 bal, 4+M (16 is probably 5m4M22 and not opening 1N)]
    \bid{3N}[s solid m, to play (about 7-card 13 HCP to 6-card 18 HCP)]
    \bid{reverse}[concentrated 14-16, 6+m and 4+ suit, NF]
    \bid{jump \& jump reverse}[inv+, spl]
    \bid{double jump}[void spl (4m = 6+m, 4+fit)]
    \bid{1C - 1D - 1H \& 1D - 1H - 1S}[11-17, 4+S (18+ uses Gazilli)]
        \subbid{PLOB (4SF1)}
    \bid{1D - 1M - 2C}[11-15, 4+C]
    \bid{1C - 1DH - 1S \& 1D - 1HS - 1N}[min unbal or 16+]
    \bid{2H}[8-10 (because 1m - 2H = 3-7)]
        \subbid{+1}[8+]
            \subsubbid{then 2m/2M}[min unbal wo/w 3M]
            \subsubbid{others}[16+, GF (2N FF?)]
        \subbid{other}[min nat (jump = weak but shapely)]
            \subsubbid{new suit}[GF]
    \bid{1C - 1DH - 1N}[11-13 bal]
        \subbid{modified 2-way}
\end{bidsemi}

After \texttt{1C - 1SN \& 1D - 1N}, it's almost the same as natrual. After opener's reverse, \texttt{2N} is the only weak and non-GF bid (OPTIONAL).

\bin{\currfiledir/seq/1m-1X}

\subsection{after Gazzilli accepted}

Rebidding \texttt{2m \& 2M} is weak (as described previously).
The only artificial bid here is that we let \texttt{2oM} become an artificial raise.
To differentiate 16-18 and 19+ (extra), we make the former rebids \texttt{2N} as a waiting bid.

% TODO: comp?
\bin{\currfiledir/seq/Gaz}

\subsection{1m - 1X; 2N}

% TODO
\begin{bidsemi}
\bid{1m - 1M(-1); 2N -}
\bid{3C}[ask 3M, may be s/o.]
    \subbid{3M}[3M]
    \subbid{3D}[no 3M]
    \subbid{(M = S) 3H}[3S4H]
        \subsubbid{P/3M}[s/o]
        \subsubbid{other}[nat GF]
\bid{3D}[fit in opener's minor ?]
\bid{3M}[6+M, slam interest]
\bid{(M = S) 3H}[55+M, MST+]
\bid{(M = H) 3S}[44M  (why not 3C ?)]
\bid{3N/4M}[s/o  (1m - 1S(-1); 2N - 4H = 55M s/o)]
\end{bidsemi}

\subsection{PH responses}

\bin{\currfiledir/seq/PH}

\subsection{comp}

\bin{\currfiledir/seq/comp}

% double
% overcalled