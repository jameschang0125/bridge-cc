\section{1m}

% \bin{\currfiledir/seq/1m}
Again, thanks to the space preserved by dutch doubleton and transfer responses, one can see that \texttt{1C - 1DH - 1S} and \texttt{1D - 1HS - 1N} (since no min bal) is undefined.
Therefore we are allowed to use them to show strong (16+) hands, with many higher bidding spaces left. 
For example, after \texttt{1D - 1S}, since \texttt{1N} shows all strong (16+) hands, \texttt{2H, 3C, 3H} are undefined. 
Unlike major suit openings where the opener can easily have shapely hands, minors don't (and among those does, lot of them are single-suited). 
Therefore, we decide to prioritize major fits -- using jump rebids as mini-splinter. 
For normal reverses, since we have already dealt with mini-splinters, we simply leave it natrual "with-Gazilli-style": showing a shapely (6-4) hand with 14-16 concentrated HCP.


Similar to major suit Gazzilli, we are allowed to include a weak variant by rebidding \texttt{2X} after Gazzilli accepted (i.e. responder rebids \texttt{+1}).
In natrual systems, however, we usually rebid \texttt{2m} with any weak unbal hands. Therefore, we are left two seemingly unnessesary options: direct rebid of \texttt{2m} and Gazzilli then \texttt{2m}.
There may be several options for this: for example, identifying a 3-card fit (compare to \texttt{1m - 1M - 2M} may be 3-card or \texttt{1C - 1DH - 1HS} is (2)3-card); or perhaps game try (targeting \texttt{3N}) with a good (AJT or KQ) 6+ suit.
In our system, we choose to do BOTH (obviously, slightly weaker then choosing one) by agreeing:
\begin{itemize}
  \setlength\itemsep{0pt}
  \item direct rebid \texttt{2m}: 14-15, good 6+ suit
  \item jump rebid \texttt{3m}: 16-17, good 6+ suit
  \item min unbal uses the Gazzilli. If accepted, rebid \texttt{2M \& 2m} to show min unbal with or without 3-card fit
\end{itemize}
Note that there are a few side effects: first, this also tighten the range of the original \texttt{3m} rebid (from 15-17 to 16-17); 
second, rebidding \texttt{3m} becomes game-forcing (18+); 
last, strong hands cannot show 3-card fit using \texttt{2M} (note: only after Gazzilli accepted), but we don't think it's a big deal since there are plenty of spaces left, including seemingly undefined \texttt{2N}.

\subsection{resp}

Other than transfer response, we also feature a slightly different (but not uncommon) response system. 
Some players simply use jump response as weak natrual, and perhaps double jump as splinter. 
IMHO this is not very efficient because it rarely happens (imagine the last time you hold 6+S, 4 HCP against \texttt{1m} opener), 
therefore we have changed it a bit (see below). 
One noticable change (recommended by Jonky) is the "reverse Flannery" which shows 3-7 HCP and 54xx+ (usually 5-7 but can be weaker due to length or Vul), therefore \texttt{1S} response followed by \texttt{2H} shows 8+.
This synergized quite well with minor-suit Gazzilli because we are allowed to show a constructed (8-10) 54xx+ with \texttt{1C - 1H; 1S - 2H \& 1D - 1S; 1N - 2H}. 

\bin{\currfiledir/seq/resp}

\subsection{rebid}

\bin{\currfiledir/seq/1m-1M}
\bin{\currfiledir/seq/1m-1X}

\subsection{1m - 1X; 2N (Wolff + 3D fit m)}

% TODO
\bin{\currfiledir/seq/1m-1M-2N}

\subsection{after Gazzilli accepted}

% TODO
\bin{\currfiledir/seq/Gaz}

\subsection{PH responses}

\bin{\currfiledir/seq/PH}

\subsection{comp}

\bin{\currfiledir/seq/comp}

% double
% overcalled