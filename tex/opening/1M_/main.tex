\section{1M}

We play Kaplan-interchange after \texttt{1H} (\texttt{1S} shows \texttt{4-S} and non-GF values, while \texttt{1N} shows \texttt{5+S}) to cope with the rebid problem.
This should be a clear winning move against natrual system, with the cost of memorization.
Therefore, I tried my best to reduce the complexity for subsequent auctions, hope that it helps.
This is a list of major tweaks:
\begin{itemize}
    \setlength\itemsep{0pt}
    \item Kaplan-interchange: \texttt{1S} shows \texttt{4-S} and non-GF values, while \texttt{1N} shows \texttt{5+S}
    \item one exception above is that inv with 6S will bid \texttt{1S} initially, then rebid 2S regardless of opener's rebid. As a consequence (and similar to \texttt{1C - 2D = inv}), a jump rebid is GF after \texttt{1H - 1N}.
    \item rebid \texttt{2C} always shows Gazzilli. (we use 6+M as the weak variant)
    \item because of Gazzilli, jumps, reverses, and \texttt{2N} rebid shows a distributional hand.
    \item \texttt{1H - 1S - 1N} shows a balanced hand or 4S, partner can inquiry with \texttt{2C}.
    \item jump oM is limit raise.
\end{itemize}

\bin{\currfiledir/seq/resp}

\subsection{rebid}

\bin{\currfiledir/seq/1M-1SN}

Some differences are made over Kaplan-interchange:
\bin{\currfiledir/seq/1H-1SN}

\subsection{after Gazzilli accepted}

% This is an optional semi-relay aim to distinguish 16-18 and 19-21. You may simply use a more natural approach (ex: 2oM = 3+oM) instead.
% \bin{\currfiledir/seq/Gaz}
(TODO)

\subsection{1M - 2N}

(TODO)
% \bin{\currfiledir/seq/1M-2N}

\subsection{2/1}

\bin{\currfiledir/seq/1M-2X}
\bin{\currfiledir/seq/1M-2X-2M}
\bin{\currfiledir/seq/1M-2X-2Y}

\subsection{PH responses}

\bin{\currfiledir/seq/PH}

\subsection{comp}

?
% \bin{\currfiledir/seq/comp}

% double
% overcalled